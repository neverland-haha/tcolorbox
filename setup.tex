%插图
\usepackage{graphicx}
%盒子
\usepackage[most]{tcolorbox}
%设置脚注
\usepackage[noadjust]{marginnote}
\renewcommand*{\raggedleftmarginnote}{}
\renewcommand*{\raggedrightmarginnote}{}
\usepackage[marginal]{footmisc}
%特殊符号及数学符号
\usepackage{pifont}
\usepackage{amssymb,amsmath}
\usepackage{breqn}
\usepackage{pxfonts}

%设置列表环境的标号以及引用,如需设置多层级请自行参阅宏包使用手册
\usepackage{enumitem}
\setlist[enumerate,1]{label = (\arabic*),
	ref   = (\arabic*)}
%设置页面尺寸
\usepackage[left=2cm,right=2cm,top=1.5cm,bottom=1.5cm]{geometry}
%页眉页脚
\usepackage{fancyhdr} 
\pagestyle{plain}
%定义脚注
\makeatletter
\renewcommand\thefootnote{\myfootnotestyle{\arabic{footnote}}}
\def\@makefnmark{\hbox{\textsuperscript{\@thefnmark}}}
\newcommand\myfootnotestyle[1]{\ifcase#1 \or \ding{182}\or \ding{183}\or
	\ding{184}\or \ding{185}\or \ding{186}\or \ding{187}%
	\or \ding{188}\or \ding{189}\or \ding{190}\or \ding{191}\else *\fi\relax}
\makeatother

%%图文混排
\usepackage{picinpar}
%无法推导出符号定义
\newcommand{\notimplies}{%
	\mathrel{{\ooalign{\hidewidth$\not\phantom{=}$\hidewidth\cr$\implies$}}}}

%TikZ设置
\usepackage{tikz}
\usepackage[tikz]{bclogo}
\usepackage{tikzpagenodes}
\tcbuselibrary{skins}

%超链接
\usepackage[hidelinks]{hyperref}

%定义颜色
\usepackage{xcolor}
\definecolor{thinkcolor}{RGB}{227,196,144}
\definecolor{thinkcolor2}{RGB}{172,128,75}
\definecolor{observecolor}{RGB}{153,201,227}
\definecolor{explorecolor}{RGB}{178,217,200}
\definecolor{textcolor1}{RGB}{77,174,234}

\tcbset{
    common/.style={
		enhanced,
		arc=0mm,
		fonttitle=\large\bfseries,
		coltitle=black,
		attach boxed title to top left={xshift=0mm,
										yshift=-0.50mm},
		boxed title style={
			skin=enhancedfirst jigsaw,
			size=small,
			arc=5mm,
			bottom=0mm,
			left=8mm,
			right=26mm,
			top=1mm},
			boxrule=0pt,
			frame hidden},
    thinkstyle/.style={
		common,
		colbacktitle=thinkcolor,
		colframe=thinkcolor,
		colback=thinkcolor!40,
		borderline north={4pt}{0pt}{thinkcolor}},
	observestyle/.style={
		common,
		colbacktitle=observecolor,
		colframe=observecolor,
		colback=observecolor!40,
		borderline north={4pt}{0pt}{observecolor}},
	explorestyle/.style={
		common,
		colbacktitle=explorecolor,
		colframe=explorecolor,
		colback=explorecolor!40,
	borderline north={4pt}{0pt}{explorecolor}},
	before upper={\parindent1em}		%设置段首缩进
}	
%颜色
\newtcolorbox{think0}{thinkstyle,title= \raisebox{-0.1cm} { \hspace{-0.35cm} \includegraphics{logo1}} \  \raisebox{0.1cm} { \color{olive} \zihao{4} 思 \ 考} } 
\newtcolorbox{think}{thinkstyle,title=     \raisebox{-0.2cm}{\bcquestion }   \color{olive} \zihao{4} 思 \ 考 }  
\newtcolorbox{observe0}{observestyle,title= \raisebox{-0.15cm}{\hspace{-0.2cm} \includegraphics{logo2}} \  \color{yellow}  \zihao{4} 观\ 察}
\newtcolorbox{observe}{observestyle,title=   \raisebox{-0.2cm}{\bcfleur}  \color{yellow}  \zihao{4}  \ 观 \ 察}
\newtcolorbox{explore0}{explorestyle,title = \raisebox{-0.15cm} {\includegraphics{logo3} \  \raisebox{0.15cm}{
	\color{teal} \zihao{4}	 探\ 索}}}
\newtcolorbox{explore}{explorestyle,title =    \raisebox{-0.2cm}{ \bclampe}  \ \color{teal}  \zihao{4} 探 \ 索}

\newtcolorbox{custom}[2][gray]{
	common,
	title=#2,
	colbacktitle=#1,
	colframe=#1,
	colback=#1!40,
	borderline north={4pt}{0pt}{#1}}